\newacronym{ipcc}{IPCC}{Intergovernmental Panel on Climate Change}
\newacronym{un}{UN}{United Nations}
\newacronym{sdg}{SDGs}{Sustainable Development Goals}
\newacronym{ep}{EP}{Environmental Performance}
\newacronym{ee}{EE}{Energy Efficiency}
\newacronym{et}{ET}{Emerging Technologies}
\newacronym{eh}{EH}{Energy Harvesting}
\newacronym{soc}{SoC}{System on Chip}
\newacronym{ml}{ML}{Machine Learning}
\newacronym{lpwan}{LPWAN}{Low-Power Wide-Area Network}
\newacronym{coap}{CoAP}{Constrained Application Protocol}
\newacronym{mqtt-sn}{MQTT-SN}{Message Queuing Telemetry Transport for Sensor Networks}

\newacronym{rfid}{RFiD}{Radio-frequency identification}
\newacronym{crfid}{CRFiD}{Computational Radio-frequency identification}
\newacronym{rom}{ROM}{Read-Only Memory}
\newacronym{ram}{RAM}{Read-Access Memory}
\newacronym{fram}{FRAM}{Ferroelectric Read-Access Memory}
\newacronym{eeprom}{EEPROM}{Electrically Erasable Programmable Read-Only Memory}
\newacronym{ciot}{CIoT}{Collaborative Internet of Things}
\newacronym{ar}{AR}{Activity Recognition}
\newacronym{ghm}{GHM}{Greenhouse Monitoring}
\newacronym{mems}{MEMS}{Micro-Electromechanical Systems}


\newglossaryentry{sustainability}{
  name={Sustainability},
  description={A holistic approach that considers environmental, social, and economic dimensions to meet present needs without compromising the ability of future generations to meet their own needs.\cite{unep_climatechangereport2023}}
}

\newglossaryentry{energyEfficiency}{
  name={Energy Efficiency},
  description={The goal to reduce the amount of energy required to provide products and services, by using less energy to achieve the same or an improved level of performance.}
}

\newglossaryentry{energyResilience}{
  name={Energy Resilience},
  description={The ability of energy systems to withstand, adapt to, and recover from disturbances while ensuring stable and reliable energy supply.\cite{unep_climatechangereport2023}}
}

\newglossaryentry{renewableenergy}
{
    name={Renewable Energy},
    description={Energy from sources that are naturally replenishing but flow-limited. Renewable resources are virtually inexhaustible in duration but limited in the amount of energy that is available per unit of time. These resources, such as sunlight, wind, rain, tides, waves, and geothermal heat, play a crucial role in sustainable energy systems and are central to reducing greenhouse gas emissions and combating climate change.\cite{unep_climatechangereport2023}}
}

\newglossaryentry{dpm}{
  name={DPM},
  description={Dynamic Power Management, a strategy for optimizing the power consumption of computing devices dynamically in response to workload demands and energy availability}
}

\newglossaryentry{lphd}{
  name={Low-Power Hardware Design},
  description={The design approach for creating hardware components that consume minimal power, essential for batteryless and energy-constrained devices}
}

\newglossaryentry{easd}{
  name={Energy-Aware Software Design},
  description={Software design methodology that prioritizes energy efficiency, aiming to reduce the energy consumption of software operations}
}

\newglossaryentry{ehi}{
  name={Energy Harvesting Integration},
  description={The incorporation of energy harvesting mechanisms into electronic systems, enabling them to capture and utilize ambient energy sources}
}

\newglossaryentry{ies}{
  name={Intermittent Energy Supply},
  description={Energy supply characterized by irregular availability, common with renewable energy sources like solar and wind}
}

\newglossaryentry{sopt}{
  name={Sensor Optimization},
  description={The process of enhancing sensor performance and efficiency, balancing accuracy, and resource consumption}
}

\newglossaryentry{adaptivealg}{
  name={Adaptive Algorithms},
  description={Algorithms that modify their operation or behavior in response to changes in their environment or input data}
}

\newglossaryentry{selflearningalg}{
  name={Self-Learning Algorithms},
  description={Algorithms capable of autonomously improving their performance over time through experience without explicit programming for all scenarios}
}

\newglossaryentry{ambientenergy}{
  name={Ambient Energy Sources},
  description={Environmental energy sources, such as solar, thermal, or kinetic energy, that can be harvested to power electronic devices}
}

\newglossaryentry{energyharvesting}{
  name={Energy Harvesting},
  description={The process of capturing energy from ambient sources and converting it into usable electrical power}
}

\newglossaryentry{pmc}{
  name={Power Management Circuit},
  description={Circuits designed to manage and optimize the power usage of electronic devices, crucial for energy efficiency}
}

\newglossaryentry{ulpc}{
  name={Ultra-Low-Power Components},
  description={Electronic components engineered to operate with extremely low power consumption, suitable for energy-harvesting applications}
}

\newglossaryentry{est}{
  name={Energy Storage Technologies},
  description={Technologies used for the storage of energy, such as batteries and \gls{supercap}, particularly important for managing intermittent energy supplies}
}

\newglossaryentry{supercap}{
  name={Supercapacitor},
  description={A high-capacity capacitor with capacitance values much higher than other capacitors but lower voltage limits, which bridges the gap between electrolytic capacitors and rechargeable batteries. It stores energy through a static mechanism, which allows for rapid charging and discharging cycles compared to batteries}
}

\newglossaryentry{poweraware}{
  name={Power Aware},
  description={Refers to the system's capability to monitor and adapt its power consumption in real-time, aiming to optimize power usage and performance without compromising the operational needs of the system.}
}

\newglossaryentry{energyaware}{
  name={Energy Aware},
  description={Involves understanding and managing the overall energy consumption of a system over a period, focusing on achieving long-term energy efficiency and sustainability by optimizing energy usage patterns and behaviors.}
}

\newglossaryentry{process}{
  name={Process},
  description={In computing, a process is an instance of a computer program that is being executed. It contains the program code and its current activity.}
}

\newglossaryentry{eveh}{
    name={EVEH},
    description={The Electromagnetic Vibration Energy Harvesters, or EVEH, were proposed in the late 1900s. These transducers transform kinetic energy (vibration) into electrical energy.}
} 

\newglossaryentry{task}{
  name={Tasks},
  description={Tasks can be considered as smaller than \gls{process}, more specific units of work executed by the computer.}
}

\newglossaryentry{fsm}{
  name={Finite State Machines (FSM)},
  description={are computational models used to design logic in systems. They transition from one state to another based on certain inputs and a set of rules. FSMs are particularly useful in ultra-low power, battery-less IoT devices because they can efficiently manage device operations with minimal power usage. By simplifying the control logic into defined states, FSMs can help reduce the power consumed by these devices during both active and idle periods.}
}

\newglossaryentry{electrostatic}{
    name={Electrostatic},
    description={Electrostatic phenomena involve the interactions and forces between stationary electric charges, explained by Coulomb's law, which includes the forces that electric charges exert on each other without moving.}
}

\newglossaryentry{piezoelectric}{
    name={Piezoelectric},
    description={It generally refers to the ability of certain materials to generate an electric charge in response to applied mechanical stress. This effect is reversible, meaning that these materials can also change shape when an electric field is applied, a phenomenon used in various applications such as sensors, actuators, and generators\cite{enwiki:1212461326}.}
}

\newglossaryentry{electromagnetic}{
    name={Electromagnetic},
    description={It encompasses the laws and phenomena associated with the electromagnetic force, which is a fundamental interaction between particles with electric charges. The electromagnetic force is responsible for practically all the phenomena encountered in daily life (with the exception of gravity), including the technologies we use, from household appliances to understanding the structure of atoms\cite{enwiki:1210790681}.}
}

\newglossaryentry{magnetorestrictive}{
    name={MagnetoRestrictive},
    description={Similar to piezoelectric materials, magnetostrictive materials change their shape or dimensions in the presence of a magnetic field. This property is utilized in various applications such as sensors and actuators, where the conversion between magnetic and mechanical energy is required. While a direct source definition was not retrieved in this search, this explanation aligns with the known scientific understanding of the term.}
}






\newglossaryentry{rehash}{
    name={REHASH},
    description={is a framework, is a flexible heuristic-adaptation-based runtime for intermittently-powered devices.\cite{gitlab:rehash}},
    first={REHASH}
}

\newglossaryentry{mnist}{
    name={MNIST},
    description={is a framework, is a flexible heuristic-adaptation-based runtime for intermittently-powered devices.\cite{gitlab:rehash}},
    first={MNIST}
}

\newglossaryentry{heuristic}{
    name={Heuristic},
    description={is a function developed-defined logic statement that takes an equation composed of measured signals and calculates a binary outcome, i.e. adapt up, or adapt down. The outcome of the heuristic function decides when to adapt tasks and knobs},
    first={Heuristic}
}


\newglossaryentry{allornothingsemantics}{
    name={All-Or-Nothing semantics},
    description={refers to a principle or approach where an action is executed in its entirety or not at all. In other words, if any part of the action fails or encounters an issue, the entire action is considered unsuccessful, and no changes are made to the system. This ensures that data remains in a consistent state and prevents partial changes that could lead to inconsistencies or errors. e.g: Transaction on Database},
    first={All-Or-Nothing semantics}
}

\newglossaryentry{atomizeprocesses}{
    name={Atomizing processes},
    description={is the application of the \textit{Divide and Conquer} principle at a lower level. In this approach, tasks are broken down into subtasks that remain indivisible. If one of these subtasks fails, the entire process becomes stuck until that specific action succeeds. This ensures that the process maintains its integrity and consistency throughout its execution.\cite{enwiki:1165389216}},
    first={Atomizing processes}
}

\newglossaryentry{nonVolatileMemory}{
    name={Non-Volatile Memory},
    description={is a type of memory where \textbf{data persists even in the absence of power}. Examples of NVM include \gls{rom}, flash drives, \gls{fram}, \gls{eeprom} and other storage devices.},
    first={Non-Volatile Memory}
}

\newglossaryentry{volatileMemory}{
    name={volatileMemory},
    description={yis a type of memory utilized for current tasks within a running session. This form of memory offers faster data access but is temporary in nature, as data is \textbf{lost once power is disconnected}. \gls{ram} is a prime example.},
    first={Volatile Memory}
}

\newglossaryentry{fault tolerance}{
    name={fault tolerance},
    description={The ability of a system to continue operating without interruption when one or more of its components fail}
}

\newglossaryentry{cryptographic algorithms}{
    name={cryptographic algorithms},
    description={Mathematical algorithms used to secure data against unauthorized access or modification}
}

\newglossaryentry{blockchain}{
    name={blockchain},
    description={A distributed database that is used to maintain a continuously growing list of records, called blocks, which are linked and secured using cryptography}
}

\newglossaryentry{electronic waste}{
    name={electronic waste},
    description={Discarded electrical or electronic devices. Used electronics which are destined for reuse, resale, salvage, recycling, or disposal are also considered electronic waste}
}


%\newacronym{iot}{IoT}{Internet of Things}

\newglossaryentry{iot}
{
    name={Internet of Things (IoT)},
    description={concept involves devices equipped with sensor(s) that are connected to a network, enabling the digitalization of the physical world through continuous monitoring. \cite{IoT_Definition}},
    first={Internet of Things (IoT)}
}


\newglossaryentry{batteryless}
{
    name={Batteryless},
    description={is a device is one that can operate without a battery, relying instead on energy spikes to charge its capacitor to activate. This process enables the device to wake up and operate for a brief period, typically a few milliseconds.},
    first={batteryless}
}


\newglossaryentry{terraswam}{
    name={TerraSwarm},
    description={TerraSwarm is a research initiative that focuses on developing technologies related to smart dust and the \gls{iot}. It aims to create a "swarm" of smart devices that can sense and interact with the physical world in real-time. The term "terra" in TerraSwarm represents the connection to the Earth's environment. TerraSwarm's research encompasses various areas such as system architecture, energy efficiency, security, and scalability to enable the deployment of large-scale networks of interconnected smart devices. This project is ended from the 31 December 2017\cite{terraSwamWebsite}.}
}

\newglossaryentry{smartdust}{
    name={Smart Dust},
    description={referes to tiny, autonomous wireless sensors that are about the size of a grain of dust. These sensors are equipped with processing, communication, and environmental sensing capabilities, such as temperature, humidity, light sensors, and more. They are designed to be deployed in the environment to monitor and collect data on various phenomena. These sensors are typically interconnected in ad hoc networks, allowing them to communicate with each other and transmit the collected data to base stations or other devices for analysis. The idea behind Smart Dust is to create a large-scale network of distributed sensors that can be used for monitoring extensive areas, such as industrial environments, urban areas, natural habitats, etc\cite{smartDust}.}
}

\newglossaryentry{photovoltaic}{
    name={Photovoltaic},
    description={refers to the conversion of light into electricity using semiconducting materials that exhibit the photovoltaic effect. This process is a method for generating electric power by using solar cells to convert energy from the sun directly into a flow of electrons by the physical and chemical phenomenon known as the photovoltaic effect.}
}