\section{Other Challenges}

    \subsection{Energy-Efficient Communication Protocols}
    
        \paragraph{}
        The proliferation of \gls{iot} technology underscores the necessity for energy-efficient communication protocols crucial for the sustainability and operational longevity of \gls{iot} devices, especially in power-limited applications\cite{iotSustainableEnergySystems}.
        
        \paragraph{}
        The primary challenge stems from the inherent energy constraints of \gls{iot} devices and the need for constant data transmission.
        
        \paragraph{}
        Innovations include the development of \gls{lpwan} and advancements in protocols like \gls{coap} and \gls{mqtt-sn} designed to reduce power consumption\cite{SABOVIC2023100736}.
        
        \paragraph{}
        Challenges include maintaining high data transmission reliability while minimizing energy use and standardizing protocols across the \gls{iot} ecosystem.

    \subsection{Environmental Adaptability}
    
        \paragraph{}
        The effectiveness of \gls{iot} and energy harvesting technologies is influenced by their adaptability to environmental conditions\cite{iotSustainableEnergySystems}.
        
        \paragraph{}
        Environmental conditions can affect the efficiency of energy harvesters and device performance.
        
        \paragraph{}
        Efforts focus on creating resilient technologies adaptable to various conditions and integrating adaptive algorithms.
        
        \paragraph{}
        Main limitations include the complexity of designing adaptable devices without increasing costs or compromising efficiency.

    \subsection{Fault Tolerance}
    
        \paragraph{}
        The reliability of energy harvesting systems and \gls{iot} devices is critical, especially in remote or critical applications where maintenance is challenging. Fault tolerance—the ability to continue operation in the event of a component failure—is a key aspect of this reliability.
        
        \paragraph{}
        Variations in energy availability and environmental conditions can lead to unpredictable system behavior, while physical damage and component wear over time may also induce failures\cite{LIU2023113436}.
        
        \paragraph{}
        Techniques to enhance fault tolerance include redundancy, where critical components are duplicated to take over in case of failure, and the development of self-healing materials and circuits that can automatically repair minor damages.
        
        \paragraph{}
        Implementing fault tolerance increases the complexity and cost of devices. Moreover, there is a trade-off between the level of redundancy and the device's energy consumption and size.

    \subsection{Security and Privacy}
    
        \paragraph{}
        As \gls{iot} devices often collect and transmit sensitive data, ensuring their security and privacy is paramount. The challenge is magnified in energy harvesting \gls{iot} devices, where resource limitations may restrict the implementation of robust security measures.
        
        \paragraph{}
        The open and distributed nature of \gls{iot} networks, along with the use of standard communication protocols, can expose devices to various security threats, including data breaches and unauthorized access.
        
        \paragraph{}
        Developing lightweight cryptographic algorithms and secure communication protocols specifically designed for \gls{iot} environments. Efforts also include the use of blockchain technology to ensure data integrity and authentication\cite{iotSecurity}.
        
        \paragraph{}
        Achieving a balance between security level and energy consumption remains a significant challenge, as more sophisticated security measures typically require additional computational resources.

    \subsection{Lifecycle Management}
    
        \paragraph{}
        The environmental impact of \gls{iot} devices, particularly those enabled by energy harvesting technologies, extends beyond their operational life. Effective lifecycle management, including recycling and end-of-life disposal, is essential for minimizing this impact.
        
        \paragraph{}
        The increasing volume of electronic waste, coupled with the hazardous materials often used in electronic components, poses a significant environmental threat\cite{iotSustainableEnergySystems}.
        
        \paragraph{}
        Strategies include the design of devices with recyclable materials, the development of biodegradable electronic components, and the implementation of take-back programs for electronic waste.
        
        \paragraph{}
        The primary challenges are the cost and logistical complexity of recycling programs, as well as the current lack of standardization for the recyclability of \gls{iot} devices.
